\documentclass[11pt]{report}
% PACKAGES
\usepackage{amsmath, amsfonts, amssymb}
\usepackage{color}
\usepackage[dvips]{graphicx}
\usepackage[utf8]{inputenc}
\usepackage[brazil]{babel}

\setlength{\headsep}{0cm}
% DIST\^{A}NCIA TEXTO/CABE\c{C}ALHO

\setlength{\textwidth}{18cm}
% LARGURA DO TEXTO

\setlength{\textheight}{22.5cm}
% ALTURA DO TEXTO

\setlength{\oddsidemargin}{-1cm}
% MARGEM \'{I}MPAR

\setlength{\evensidemargin}{0cm}
% MARGEM PAR

\setlength{\topmargin}{0cm}
% MARGEM SUPERIOR

\renewcommand{\baselinestretch}{1.1}
% DIST\^{A}NCIA ENTRE LINHAS

\pretolerance=10000

\begin{document}
	\thispagestyle{empty}

	\begin{center}
	Escreva esse texto em modo centralizado, \textit{deixando essa parte em itálico} e \textbf{essa parte em negrito}.
	\end{center}
	
	\hspace*{\fill}Escreva esse texto no lado esquerdo, com essa {\huge palavra} em tamanho huge. \\
	
    Comprove que sabe iniciar um novo parágrafo e escreva a seguinte equação enumerada:
    \begin{equation}
    f(x)=
    \left 
    \{
    \begin{array}{ccl}
    x^3 + \frac{5}{x} & \mbox {se} & x \leq 0 \\
    -\sum\limits_{n=1}^{\infty} \frac{e^{nx}}{n!} & \mbox{se} & 0 < x \leq 1
    \end{array}
    \right. \label{eq:1}
    \end{equation}

	\begin{flushleft}
		Agora vamos fazer enumerações em cadeia
		\begin{enumerate}
			\item Primeiro item e seus subitens:
			\begin{enumerate}
				\item [i)] subitem i.
				\item [ii)] subitem ii.
			\end{enumerate}
			\item Segundo item.
		\end{enumerate}
	
		Agora comprovo aqui que sei mencionar uma equação enumerada: basta observar a equação \eqref{eq:1}, por exemplo.
		\\
		Inicio abaixo uma seção.
	\end{flushleft}

\end{document}